\documentclass[a4paper,11.5pt]{article}

\usepackage[textwidth=135mm]{geometry}

\usepackage[utf8]{inputenc}

\usepackage[T1]{fontenc}

\PassOptionsToPackage{defaults=hu-min}{magyar.ldf}

\usepackage[magyar]{babel}

\usepackage{amsmath,amssymb,paralist,array}



\begin{document}

	%%%%%%%%%%%RÖVIDÍTÉSEK%%%%%%%%%%

	%\setlength\parindent{0pt}

	\def\a{\textbf{a}}

	\def\b{\textbf{b}}

	\def\N{\hskip 10 true mm}

	\def\a{\textbf{a}}

	\def\b{\textbf{b}}

	\def\c{\textbf{c}}

	\def\d{\textbf{d}}

	\def\e{\textbf{e}}

	\def\gg{$\gamma$}

	\def\vi{\textbf{i}}

	\def\jj{\textbf{j}}

	\def\kk{\textbf{k}}

	\def\fh{\overrightarrow}

	\def\l{\lambda}

	\def\m{\mu}

	\def\v{\textbf{v}}

	\def\0{\textbf{0}}

	\def\s{\hspace{0.2mm}\vphantom{\beta}}

	\def\Z{\mathbb{Z}}
	\def\Q{\mathbb{Q}}
	\def\R{\mathbb{R}}
	\def\C{\mathbb{C}}
	\def\N{\mathbb{N}}
	\def\Rn{\mathbb{R}^{n}}

	%%%%%%%%%%%%%%%%%%%%%%%%%%%%%%%%%


	\begin{flushright}

		Husi kidolgozása (\LaTeX)

	\end{flushright}

	\begin{center}

		Lineáris algebra (A, B, C)\\

	1. előadás

	\end{center}

	\noindent Kezdjük az alapfogalmak tisztázásával:
	\begin{compactitem}
		\item $\forall$ - univerzális kvantor, köznapi nyelven ,,minden''
		\item $\exists$ - egzisztenciális kvantor, köznapi nyelven ,,létezik/van olyan''
		\item $!$ \ - egyértelműség jele, ,,csak egy ilyen van''
	\end{compactitem}
	Ezeknek a jeleknek a megértése kulcsfontosságú a későbbi anyagok megértésénél, így érdemes gondoskodni arról, hogy teljesen tisztában legyünk jelentésükkel.
	
	Alkalmazásban: $\forall a \in \R~\exists!b \in \R: \quad a+b=0.~$ Ez köznapi nyelven azt jelenti, hogy \emph{minden} $a$ valós számhoz \emph{egyértelműen létezik} olyan $b$ valós szám, melyre $a+b=0$. A kijelentés pedig nyilvánvalóan igaz, hisz minden valós számnak a $(-1)$-szerese lesz az, mellyel összeadva az összeg 0. A tételek helyes kiolvasásáról bővebben a második előadásban lesz szó.\\
	DEFINÍCIÓ: (Descartes-szorzat) Legyen $A$ és $B$ halmazok. Ekkor $A$ és $B$ Descartes-szorzata:
	\begin{center}
		$A \times B=\{(a,b)~|~a\in A~~\text{és}~~b\in B\}$.
	\end{center}
	\noindent Szemléletesebben $A\times B$ olyan \emph{rendezett párok} halmaza, melynek első komponense A-béli, második komponense B-beli. A rendezet pároknak van egy fontos tulajdonsága a hagyományos halmazokkal szemben, itt a sorrend ugyanis számít, $(a,b) \overset{\text{\textit{ált.}}}{\not=} (b,a)$.
	
	\noindent Alkalmazásban: $A=\{1, 2, 3\}, \quad B=\{a, b, c\}$ esetén\\
	$A\times B=\{(1, a), (1, b), (1, c), (2, a), (2, b), (2, c), (3, a), (3, b), (3, c)\}$.
	
	\bigskip
	Tanulmányaink folyamán a valós számok Descartes-szorzatával fogunk túlnyomó többségben foglalkozni, melyeket így jelölünk röviden: $\R \times\R=\R^2,~~\overbrace{\R\times\ldots\times\R}^{n \text{ darab}}=\Rn$.
	
	Hol találkoztunk mi középiskolában rendezett párokkal (azon belül is rendezett \emph{szám}pá\-rok\-kal)? A derékszögű koordinátarendszerbéli pontok, vagy ahhoz tartozó helyvektorok felfoghatók rendezett valós szám párokként, pl. $(4, 2)$. 
	
	Elevenítsük fel a síkbeli vektorok összeadását! Derékszögű koordinátarenszdszerben a ,,rajzolós'' paralelogramma módszer ismeretes, algebrai módszerekkel a koordinátánkénti  összeadás, pl. $(1, 2) + (3, 4) =(4, 6)$. Általánoasbban, egy $(x_1, y_1) \in \R^2$ és $(x_2, y_2) \in \R^2$ koordinátákkal rendelkező vektor összege $(x_1+x_2, y_1+y_2) \in \R^2$.
	
	Skalárral való szorzás (figyelem, NEM skaláris szorzás!) vektorokra úgy értelmeztük, hogy ,,annyiszeresére zsugorítom/növelem, vagy változtatom az irányát'' a vektornak, amennyivel szorzom. Ez algebrai úton egy tetszőleges $\lambda \in \R$ számra és $(x,y)\in\R^2$ vektorra értelmezett skalárral vett szorzás végeredménye $\lambda(x, y)= (\lambda x, \lambda y) \in \R^2$.
	
	Vegyünk most egy kicsit absztraktabb példát, egy számegyenest. Felfoghatjuk a számegyenesen lévő számokat úgy, mint 1 komponensű\footnote{komponens = elem, ennél a példánál még azt is mondhatjuk hogy koordináta, de ez később gondokat okozhat, amikor ez a két fogalom nem lesz ekvivalens.} vektorok, pl. 3 helyett mondjuk azt hogy (3) az origóból a 3-ba mutató helyvektor. A paralelogramma módszer továbbra is működik, hisz ha (2) és (1) vektort össze akarjuk adni, akkor (1) kezdőpontját (2) végpontjához illesztve megkapjuk a (3) vektort. 
	
	Ebben tulajdonképpen nincs semmi meglepő. Ez csak valós számok összeadása. Megemlítendő, hogy a skalárral való szorzás is tökéletesen működik, pl. $6\cdot(4)=(24)$.
	
	Ugorjunk egy újabb példára, egy 3 dimenziós derékszögű koordinátarendszerbe. Itt minden vektornak 3 komponense van. Nem meglepő módon $(1,2,3)+(5,3,0)=(6,5,3)$, skalár szorzással $5(1,0,3)=(5,0,15)$. Minden vektor itt felfogható rendezett valós számhármasként ($\R^3$).
	
	\medskip
	Feltűnhet, hogy minden alkalommal, amikor műveletet hajtunk végre egy vektoron, simán egyszerre több számmal csináljuk ugyanazt. Ezt a tulajdonságot másra is felhasználhatnánk, ahol egyszere több számon szeretnénk hasonló műveleteket végrehajtani. Próbáljunk megalkotni egy olyan struktúrát, mely ezen igényünket kielégíti! Induljunk ki rendezett valós szám 3-asokból. (Itt már nem geometriai vektorokról fogunk beszélni!)
	
	$\begin{bmatrix}
		\alpha_1 \s\\
		\alpha_2\s \\
		\alpha_3\s \\
	\end{bmatrix} \in \R^3$ legyen egy ilyen elem, $\R^3$ jelölje az összes ilyen rendezett valós számhármas halmazát. Bár vektoroknál volt definiálva összeadás és skalárral vett szorzás, mi most valami újat szeretnénk alkotni, így itt definiálnunk kell ezeket a műveleteket is.
	
		{\centering
			$ \a=
			\begin{bmatrix}
			\alpha_{1}\s \\
			\alpha_{2}\s \\
			\alpha_{3}\s \\
			\end{bmatrix},~ \b =
			\begin{bmatrix}
			\beta_{1}\s \\
			\beta_{2}\s \\
			\beta_{3}\s \\
			\end{bmatrix} \in  \R^{3}$ \quad esetén legyen $\quad \a + \b= 
			\begin{bmatrix}
			\alpha_{1}+\beta_{1}\s \\
			\alpha_{2}+\beta_{2}\s \\
			\alpha_{3}+\beta_{3}\s \\
			\end{bmatrix} \in  \R^{3}, $
			
			$\gamma\in  \R,  \a =
			\begin{bmatrix}
			\alpha_{1}\s \\
			\alpha_{2}\s \\
			\alpha_{3}\s \\
			\end{bmatrix} \in  \R^{3}$\qquad esetén legyen $\quad\gamma\a=
			\begin{bmatrix}
			\gamma\alpha_{1}\s \\
			\gamma\alpha_{2}\s \\
			\gamma\alpha_{3}\s \\
			\end{bmatrix} \in \R^{3}$.
			\par}
	
	\noindent Bár nem \emph{muszáj} oszloposan írni, így célszerű használni a későbbi alkalmazásuk miatt.
	
	Lévén ezek csak rendezett valós számhármasok, félrevezető lenne elemeit koordinátáknak nevezni, így maradjunk a komponens (vagy elem) kifejezésnél. Ez a rész később jobban ki fog tisztulni.
	
	Figyeljük meg ezek alkalmazásait egy lineáris egyenletrendszerben (továbbiakban LER): $x, y, z \in \R $
	
	{\centering
		$\begin{matrix} 
		x + y + z &=& 3\\
		2x + 3y + 4z &=& 9\\
		3x + 4y + 5z& =& 12
		\end{matrix}$
		\par}

	\noindent Legyen:
	
	{\centering
		$\a=
		\begin{bmatrix}
		1\s \\
		2\s \\
		3\s \\
		\end{bmatrix},~
		\b~=
		\begin{bmatrix}
		1\s \\
		3\s \\
		4\s \\
		\end{bmatrix},~
		\c~=
		\begin{bmatrix}
		1\s \\
		4\s \\
		5\s \\
		\end{bmatrix},~
		\d~=
		\begin{bmatrix}
		3\s \\
		9\s \\
		12\s \\
		\end{bmatrix}$
	\par}
	
	\noindent ahol \a~$x$ együtthatóiból, \b~ $y $ együtthatóiból, \c~$z$ együtthatóiból, és \d~az egyenletek jobb oldalából áll. Ekkor a fenti LER egyetlen egyenletbe sűríthető:
	
	{\centering
		$x\a + y\b + z\c = \d$.
	\par}
	
	\noindent E kettő ekvivalens, hisz:
	
	{\centering
		$x\a + y\b + z\c = \d \Leftrightarrow 
		\begin{bmatrix}
		1\s \\
		2\s \\
		3\s \\
		\end{bmatrix}\cdot x+
		\begin{bmatrix}
		1\s \\
		3\s \\
		4\s \\
		\end{bmatrix}\cdot y+
		\begin{bmatrix}
		1\s \\
		4\s \\
		5\s \\
		\end{bmatrix}\cdot z=
		\begin{bmatrix}
		3\s \\
		9\s \\
		12\s \\
		\end{bmatrix}\Leftrightarrow$ \\ $ \Leftrightarrow
		\begin{bmatrix}
		1x\s \\
		2x\s \\
		3x\s \\
		\end{bmatrix}+
		\begin{bmatrix}
		1y\s \\
		3y\s \\
		4y\s \\
		\end{bmatrix}+
		\begin{bmatrix}
		1z\s \\
		4z\s \\
		5z\s \\
		\end{bmatrix}=
		\begin{bmatrix}
		3\s \\
		9\s \\
		12\s \\
		\end{bmatrix} \Leftrightarrow
		\begin{bmatrix} 
			x + y + z &=& 3\\
			2x + 3y + 4z &=& 9\\
			3x + 4y + 5z& =& 12
		\end{bmatrix}.
		$
		\par}
	
	\noindent Ezeket a lépéseket azért tehettük meg, mert korábban e szabályok szerint definiáltuk őket. Könnyen megállapítható hogy $\a+\b+\c=\d$, megoldás lesz, azaz $x_1=1, y_1=1, z_1=1$. Több (sőt, végtelen sok) megoldás van, azonban mi haladjunk tovább: 
	
	\pagebreak
	Próbáljuk meg a példánkat általánosítani\footnote{ide kéne valami frappáns leírás}. Meglehetősen kényelmetlen lenne a fenti LER-ben minden együtthatónak és változónak új betűjelet adni, így az alábbi jelölést alkalmazzuk majd:
	
	\begin{center}
		$\begin{matrix}
		a_{11}x_{1} + a_{12}x_{2} + a_{13}x_{3} + a_{14}x_{4} &=& b_{1}\\
		a_{21}x_{1} + a_{22}x_{2} + a_{23}x_{3} + a_{24}x_{4} &=& b_{2}\\
		a_{31}x_{1} + a_{32}x_{2} + a_{33}x_{3} + a_{34}x_{4} &=& b_{3}\\
		\end{matrix}$
	\end{center}
	
	\noindent Az itteni jelölések összezavaróak lehetnek, szóval szerepeljen egy magyarázat is:
	
	
	\begin{center}
		\setlength{\extrarowheight}{7pt}
	\begin{tabular}{cc|c|c|c|c|c|cccc}
		&{\LARGE 1}&&{\LARGE2}&&{\LARGE3}&&{\LARGE4}&&\\
		{\LARGE 1}&$a_{11}x_{1} $&$+$&$ a_{12}x_{2}$ &$+$&$ a_{13}x_{3}$ &$+$& $a_{14}x_{4}$ &$=$& $b_{1}$\\
		\hline
		{\LARGE 2}&$a_{21}x_{1} $&$+$&$ a_{22}x_{2}$ &$+$&$ a_{23}x_{3}$ &$+$&$ a_{24}x_{4}$ &$=$& $b_{2}$\\
		\hline
		{\LARGE 3}&$a_{31}x_{1}$ &$+$& $a_{32}x_{2}$ &$+$&$ a_{33}x_{3} $&$+$&$ a_{34}x_{4}$ &$=$& $b_{3}$\\
	\end{tabular}
	\end{center}
	
	\noindent Félreértést elkerületndő, $a_{11}$ nem a 11. $a$-ra utal, hanem az első sor első oszlopában lévő változó együtthatójára (helyes kiejtés: ,,a egy egy''). Általánosan:
	
	
	\begin{center}
		$a_{ij}x_j \rightarrow$ $i$-edik sor, $j$-edik elemének együtthatója és változója.
	\end{center}
	

	\noindent (értelemszerűen $i,j \in \N$, ezt nem sokszor fogom kiírni). Ez a kétindexes jelölés majd a mátrixoknál lesz igazán fontos. Így:
	\begin{center}
		$\b =
		\begin{bmatrix}
		b_{1}\s \\
		b_{2}\s \\
		b_{3}\s \\
		\end{bmatrix},~
		\a_{1} =
		\begin{bmatrix}
		a_{11}\s \\
		a_{21}\s \\
		a_{31}\s \\
		\end{bmatrix},~
		\a_{2} =
		\begin{bmatrix}
		a_{12}\s \\
		a_{22}\s \\
		a_{32}\s \\
		\end{bmatrix},~
		\a_{3} =
		\begin{bmatrix}
		a_{13}\s \\
		a_{23}\s \\
		a_{33}\s \\
		\end{bmatrix},~
		\a_{4} =
		\begin{bmatrix}
		a_{14}\s \\
		a_{24}\s \\
		a_{34}\s \\
		\end{bmatrix}$.
	\end{center}
	
	\noindent Ezeknek a jelöléseknek a megértése kulcsfontosságú, így erősen javallott alaposan átnézni és értelmezni őket továbbhaladás előtt.
	
	\bigskip
	Nyilván, nem csak 3 egyenlettel kell foglalkoznunk egyszerre, általában többel, mondjuk $n$ darabbal ($n \in \N$, ez is ritkán lesz kikötve):
	\begin{center}
		$\begin{matrix}
		a_{11}x_{1} + a_{12}x_{2} + a_{13}x_{3} + a_{14}x_{4} &=& b_{1}\\
		a_{21}x_{1} + a_{22}x_{2} + a_{23}x_{3} + a_{24}x_{4} &=& b_{2}\\
		\vdots\\
		a_{n1}x_{1} + a_{n2}x_{2} + a_{n3}x_{3} + a_{n4}x_{4} &=& b_{n}\\
		\end{matrix}$
	\end{center}
	\noindent Logikus lenne az általánosítás:
	\begin{center}
		$\a_1=
		\begin{bmatrix}
		a_{11}\s \\
		a_{21}\s \\
		\vdots\\
		a_{n1}\s \\
		\end{bmatrix},~\a_2
		\begin{bmatrix}
		a_{12}\s \\
		a_{22}\s \\
		\vdots\\
		a_{n2}\s \\
		\end{bmatrix},~\a_3=
		\begin{bmatrix}
		a_{13}\s \\
		a_{23}\s \\
		\vdots\\
		a_{n3}\s \\
		\end{bmatrix},~\a_4=
		\begin{bmatrix}
		a_{14}\s \\
		a_{24}\s \\
		\vdots\\
		a_{n4}\s \\
		\end{bmatrix},~\b=
		\begin{bmatrix}
		b_{1}\s \\
		b_{2}\s \\
		\vdots\\
		b_{n}\s \\
		\end{bmatrix}$.
	\end{center}
	Igen, ezeket azonban így nem adhatjuk össze, hisz egyszerre $n$ szám ($n \not=1, 3$) összeadása nem volt még definiálva. Tehát, legyenek a fenti szimbólumok rendezett valós szám $n$-esek, $\Rn$~elemei, és:
	
	\pagebreak
	 DEFINÍCIÓ: { Legyen} $n$ { rögzített pozitív
		egész szám, továbbá}
	
	{\centering 
		$\begin{bmatrix}
		\alpha_{1}\s \\
		\alpha_{2}\s \\
		\vdots\\
		\alpha_{n}\s \\
		\end{bmatrix},~
		\begin{bmatrix}
		\beta_{1}\s \\
		\beta_{2}\s \\
		\vdots\\
		\beta_{n}\s \\
		\end{bmatrix} \in\Rn\quad$ {esetén legyen}
		$\quad
		\begin{bmatrix}
		\alpha_{1}\s \\
		\alpha_{2}\s \\
		\vdots\\
		\alpha_{n}\s \\
		\end{bmatrix} +
		\begin{bmatrix}
		\beta_{1}\s \\
		\beta_{2}\s \\
		\vdots\\
		\beta_{n}\s \\
		\end{bmatrix} =
		\begin{bmatrix}
		\alpha_{1}+\beta_{1}\s \\
		\alpha_{2}+\beta_{2}\s \\
		\vdots\\
		\alpha_{n}+\beta_{n}\s \\
		\end{bmatrix} \in \Rn, $
		
		$\lambda \in  \R,
		\begin{bmatrix}
		\alpha_{1}\s \\
		\alpha_{2}\s \\
		\vdots\\
		\alpha_{n}\s \\
		\end{bmatrix} \in  \Rn\quad$ {esetén legyen}$\quad
		\lambda
		\begin{bmatrix}
		\alpha_{1}\s \\
		\alpha_{2}\s \\
		\vdots\\
		\alpha_{n}\s \\
		\end{bmatrix} =
		\begin{bmatrix}
		\lambda\alpha_{1}\s \\
		\lambda\alpha_{2}\s \\
		\vdots\\
		\lambda\alpha_{n}\s \\
		\end{bmatrix} \in  \Rn$. 
		\par}
	
	\noindent Két rendezett valós szám $n$-es akkor egyenlő, ha a két oszlop megfelelő helyein elemük megegyezik. Így a fenti, általánosított alakra is teljesül: $\a_1x_1+\a_2x_2+\a_3x_3+\a_4x_4=\b$. A gond a következő, szeretnénk, ha a fent definiált 2 művelet rendelkezne a valós számokra értelmezett összeadás és (skalár) szorzás egyes tulajdonságaival. Ilyen például:
	\begin{itemize}
		\item kommutativitás, valós számoknál: $\forall a,b \in \R: \quad a+b=b+a$ (felcserélhetőség),
		\item asszociativitás, valósoknál: $\forall a,b,c \in \R: \quad (a+b)+c=a+(b+c)$ (szabadon zárójelezhetőség)
	\end{itemize}
	
	\noindent Így elérkeztünk az első tételünkhöz:
	
	 TÉTEL: A rendezett valós szám $n$-esek közötti összeadás kommutatív.
	
	\noindent Bizonyítás:
	
	{\centering
	$\begin{bmatrix}
	\alpha_{1}\s \\
	\alpha_{2}\s \\
	\vdots\\
	\alpha_{n}\s \\
	\end{bmatrix}$,$
	\begin{bmatrix}
	\beta_{1}\s \\
	\beta_{2}\s \\
	\vdots\\
	\beta_{n}\s \\
	\end{bmatrix} \in  \Rn$~esetén  
		$\begin{bmatrix}
		\alpha_{1}\s \\
		\alpha_{2}\s \\
		\vdots\\
		\alpha_{n}\s \\
		\end{bmatrix}+
		\begin{bmatrix}
		\beta_{1}\s \\
		\beta_{2}\s \\
		\vdots\\
		\beta_{n}\s \\
		\end{bmatrix}=
		\begin{bmatrix}
		\alpha_{1}+\beta_{1}\s \\
		\alpha_{2}+\beta_2\s \\
		\vdots\\
		\alpha_{n}+\beta_n\s \\
		\end{bmatrix}=
		\begin{bmatrix}
		\beta_1+\alpha_{1}\s \\
		\beta_2+\alpha_{2}\s \\
		\vdots\\
		\beta_n+\alpha_{n}\s \\
		\end{bmatrix}=
		\begin{bmatrix}
		\beta_{1}\s \\
		\beta_{2}\s \\
		\vdots\\
		\beta_{n}\s \\
		\end{bmatrix}+
		\begin{bmatrix}
		\alpha_{1}\s \\
		\alpha_{2}\s \\
		\vdots\\
		\alpha_{n}\s \\
		\end{bmatrix}$.
		\par}
	
	\noindent Ezt a lépést azért tehetük meg, mert a valós számokról tudjuk hogy kommutatívak összadásra, így $\alpha_i+\beta_i=\beta_i+\alpha_i.~\blacksquare$
	
	\noindent Ezzel az első tételünket bebizonyítottuk, az összeadás (legalábis 2 elemre) kommutatív.
	
	 TÉTEL: ~$\a,\b,\c \in \Rn$-re reljesül, hogy $(\a+\b)+\c=\a+(\b+\c)$.
	
	\noindent Bizonyítás: 
	
	{\centering
		$\begin{bmatrix}
		\alpha_{1}\s \\
		\alpha_{2}\s \\
		\vdots\\
		\alpha_{n}\s \\
		\end{bmatrix}$,$
		\begin{bmatrix}
		\beta_{1}\s \\
		\beta_{2}\s \\
		\vdots\\
		\beta_{n}\s \\
		\end{bmatrix}$,$
		\begin{bmatrix}
		\gamma_{1}\s \\
		\gamma_{2}\s \\
		\vdots\\
		\gamma_{n}\s \\
		\end{bmatrix} \in  \Rn$~esetén $
		\begin{pmatrix}
		\begin{bmatrix}
		\alpha_{1}\s \\
		\alpha_{2}\s \\
		\vdots\\
		\alpha_{n}\s \\
		\end{bmatrix}+
		\begin{bmatrix}
		\beta_{1}\s \\
		\beta_{2}\s \\
		\vdots\\
		\beta_{n}\s \\
		\end{bmatrix}
		\end{pmatrix}+
		\begin{bmatrix}
		\gamma_{1}\s \\
		\gamma_{2}\s \\
		\vdots\\
		\gamma_{n}\s \\
		\end{bmatrix}$=$
		\begin{bmatrix}
			\alpha_{1}+\beta_{1}\s \\
			\alpha_{2}+\beta_2\s \\
			\vdots\\
			\alpha_{n}+\beta_n\s \\
		\end{bmatrix}+
		\begin{bmatrix}
			\gamma_{1}\s \\
			\gamma_{2}\s \\
			\vdots\\
			\gamma_{n}\s \\
		\end{bmatrix}=
		\begin{bmatrix}
			(\alpha_{1}+\beta_1)+\gamma_1\s \\
			(\alpha_{2}+\beta_2)+\gamma_2\s \\
			\vdots\\
			(\alpha_{n}+\beta_n)+\gamma_n\s \\
		\end{bmatrix}$=$
		\begin{bmatrix}
			\alpha_{1}+(\beta_1+\gamma_1)\s \\
			\alpha_{2}+(\beta_2+\gamma_2)\s \\
			\vdots\\
			\alpha_{n}+(\beta_n+\gamma_n)\s \\
		\end{bmatrix}=
		\begin{bmatrix}
		\alpha_{1}\s \\
		\alpha_{2}\s \\
		\vdots\\
		\alpha_{n}\s \\
		\end{bmatrix}+
		\begin{bmatrix}
			\beta_1+\gamma_1\s \\
			\beta_2+\gamma_2\s \\
			\vdots\\
			\beta_n+\gamma_n\s \\
		\end{bmatrix}= $ \linebreak $
		\begin{bmatrix}
			\alpha_{1}\s \\
			\alpha_{2}\s \\
			\vdots\\
			\alpha_{n}\s \\
		\end{bmatrix}+
		\begin{pmatrix}
			\begin{bmatrix}
				\beta_{1}\s \\
				\beta_{2}\s \\
				\vdots\\
				\beta_{n}\s \\
			\end{bmatrix}+
			\begin{bmatrix}
				\gamma_{1}\s \\
				\gamma_{2}\s \\
				\vdots\\
				\gamma_{n}\s \\
			\end{bmatrix}
		\end{pmatrix}.~\blacksquare$
		\par}
	
	\noindent Ezen a ponton érdemes megjegyezni, hogy egyszerre 3 rendezett valós szám $n$-est összeadni nem tudunk, mindig páronként tesszük ezt meg (ezzel még én is csaltam pár oldallal korábban!). Persze könnyen belátható, hogy így VÉGES sok rendezett valós szám $n$-es összeadása elvégezhető egy lépésben, de valójában itt is páronkénti összeadások sorozatát hajtjuk végre. Ne feledjük el, itt csak 3 $\Rn$-béli rendezett valós szám $n$-es asszociativitását bizonyítottuk, nem tetszőleges véges sokét. Nemsokára ezt is belátjuk, azonban ez kicsit nagy falat lenne ilyen hamar.
	
	Valós számok összeadásánál volt un. semleges elemünk. Olyan szám, melyre teljesül, hogy minden valós számmal vett összege az eredeti számot adja vissza: nulla.
	
	Rendezett valós szám $n$-eseknél mi lesz az additív (összeadásra értelmezett) semleges elem? Könnyen belátható:
	
	\begin{center}
		$\begin{bmatrix}
		\alpha_{1}\s \\
		\alpha_{2}\s \\
		\vdots\\
		\alpha_{n}\s \\
	\end{bmatrix}=
	\begin{bmatrix}
	\alpha_{1}+0\s \\
	\alpha_{2}+0\s \\
	\vdots\\
	\alpha_{n}+0\s \\
	\end{bmatrix}=
	\begin{bmatrix}
	\alpha_{1}\s \\
	\alpha_{2}\s \\
	\vdots\\
	\alpha_{n}\s \\
	\end{bmatrix}+
	\begin{bmatrix}
	0\s \\
	0\s \\
	\vdots\\
	0\s \\
	\end{bmatrix}$.
	\end{center}
	\noindent Ez a csupa nullából álló szám $n$-es hasonlóan viselkedik mint a középsikolában tanult nullvektor! Továbbiakban jelölje ezt az elemet $\0 \in \Rn$. Ugyanígy additív inverzet is kereshetünk (olyan elemet, mellyel bármely más elemmel összeadva visszakapjuk a semleges elemet, pl.: $a \in \R: (-a)+a=0$, Itt $a$ additív inverze $(-a)$):
	
	\begin{center}
		$\a=
	\begin{bmatrix}
	\alpha_{1}\s \\
	\alpha_{2}\s \\
	\vdots\\
	\alpha_{n}\s \\
	\end{bmatrix} \in \Rn: \quad
	\begin{bmatrix}
	0\s \\
	0\s \\
	\vdots\\
	0\s \\
	\end{bmatrix}=
	\begin{bmatrix}
	\alpha_{1}+(-\alpha_1)\s \\
	\alpha_{2}+(-\alpha_2)\s \\
	\vdots\\
	\alpha_{n}+(-\alpha_n)\s \\
	\end{bmatrix}=
	\begin{bmatrix}
	\alpha_{1}\s \\
	\alpha_{2}\s \\
	\vdots\\
	\alpha_{n}\s \\
	\end{bmatrix}+
	\begin{bmatrix}
	-\alpha_{1}\s \\
	-\alpha_{2}\s \\
	\vdots\\
	-\alpha_{n}\s \\
	\end{bmatrix}=\a+(-\a).$
	\end{center}
	Jelölje \a~inverzét (vagy ellentetjét) $(-\a)$. Belátható, hogy skalárral vett szorzással\\ $(-1)\a=(-\a)$.
	
	Nagyon jó úton járunk egy kényelmesen kezelhető, a vektorok, és azon értelmezett összeadáshoz és skalárral vett szorzáshoz hasonló struktúra kialakításán. Valamennyi dolog még könnyen belátható az előző két tételhez hasonlóan: ilyen a skalárral való szorzás és összeadás kétféle disztributív kapcsolata (ez mindjárt alaposabban ki lesz fejtve) és a skalárral való szorzás ,,asszociativitása''. Ezen a ponton igen jól kezelhető rendszert kaptunk, így már bátran összefoglalhatunk:
	
	\medskip
	$(\Rn, +, \lambda \cdot )$ {tulajdonságai:}
	
	\smallskip
	\noindent I./1. $+: \Rn \times  \Rn  \rightarrow  \Rn,$ azaz
	$\forall \a, \b \in  \Rn$-hez hozzá van rendelve egy $\a + \b \in
	 \Rn$ ; 
	
	\smallskip
	\noindent I./2. $\forall \a, \b \in  \Rn  \quad \b + \a =\a +
	\b$ (kommutativitás);  
	
	\smallskip
	\noindent I./3. $\forall \a, \b, \c \in  \Rn  \quad (\a + \b )+ \c = \a +
	(\b + \c)$ (asszociativitás);
	
	\smallskip
	\noindent I./4. $\exists \0 \in   \Rn   :  \forall \a \in  \Rn  \quad
	\0+\a =\a (=\a +\0)$ (,,nullvektor'' létezése);
	
	\smallskip
	\noindent I./5. $\forall \a \in  \Rn~ \exists (-\a )\in  \Rn~   : 
	(-\a )+\a =\0~ (=\a+(-\a ))$ (,,ellentett'' létezése);
	
	\smallskip
	\noindent II./1. $\l \cdot :   \R  \times  \Rn  \rightarrow  \Rn$, azaz
	$\forall \l \in  \R,~\a \in  \Rn$-hez hozzá van rendelve egy $\l
	\a \in  \Rn$; 
	
	\smallskip
	\noindent II./2. $\forall \l, \m \in  \R , \a \in  \Rn  \quad (\l \m )\a =\l
	(\m \a )$ (,,asszociativitás'');
	
	\smallskip
	\noindent II./3. $\forall \l, \m \in  \R , \a \in  \Rn  \quad (\l +\m )\a =\l
	\a +\m \a$ (első disztibutivitás);
	
	\smallskip
	\noindent II./4. $\forall \l \in  \R , \a , \b \in  \Rn  \quad \l (\a +\b )=\l
	\a +\l \b $ (második disztributivitás);
	
	\smallskip
	\noindent II./5. $\forall \a \in  \Rn  \quad 1\a =\a $.
	
	\medskip 
	Bár elsőre ez lehet hogy nehezen érthető, de ez csupán minden, már korábban elhangzott dolog összefoglalása volt. Diszkrét matematika I-ben elhangzó algebrai struktúrával kapcsolatos tudások megszerzése után még sokkal világosabb lesz ez a rész.
	
	Lévén rendelkeznek minden középiskolában tanult vektor tulajdonsággal, $\Rn$-t \textbf{vektortér}nek, elemeit \textbf{vektor}oknak nevezzük majd (rendezett valós szám $n$-esek helyett). Tartsuk észben, itt nem ,,$n$ dimenzióbéli elképzelhetetlen irányított szakaszokkal bűvészkedünk'' -- a vektor kifejezés a hasonló tuladjonságuk miatt célszerű, illetve mert $\R^1, \R^2, \R^3$-ban szemléletesen is megmutatható valamennyi dolog. Motivációnk továbbra is a kényelmes adatkezelés, melyet e vektorokban tárolhatunk.\\
	Értelemszerűen $\Rn$-nek vannak részhalmazai, de érdemes egy speciális részhalmazt definiálni: olyat, mely összeadásra és skalárral való szorzásra zárt. Ez alatt azt értjük, hogy bármely két elemét vesszük e $W \subset \Rn$~részhalmaznak, az összegük is eleme lesz $W$-nek. Skalárral vett szorzásra \emph{bármely} valós számmal vett szorzás eredményét szintén tartalmazza $W$. Összefoglalva:
	
	 DEFINÍCIÓ: $W$ { altere} $\Rn${-nek} \quad [{ jelölése:}
	$W\leqq  \Rn$], \quad { ha teljesülnek:} 
	
	\noindent $\text{1.}\quad  \emptyset\not= W\subseteqq  \Rn  \qquad 
	\text{2.}\quad \a ,\b \in W \Rightarrow \a +\b \in W \qquad 
	\text{3.}\quad \l \in  \R , \a \in W \Rightarrow \l \a \in W .$ 
	
	\medskip
	\noindent Példa:
	
	{\centering 
		$W=\left\{
	\begin{bmatrix}
		\alpha\s\\
		0\s\\
		\vdots \\
		0\s\\
	\end{bmatrix} \in \Rn: \quad \alpha\in \R \right\}$.
	\par}
	
	\noindent Ez egy olyan halmaz, mely minden olyan vekort tartalmaz, melyeknek az első komponensétől eltekintve minden eleme 0. 
	Megjegyzés: \0 minden altérnek eleme, hisz $\lambda=0$ esetén (azaz ha skalár szorzásnál 0-val szorozzuk meg a vektort) ezt is tartalmaznia kell. Az alterek \emph{nem} kevesebb komponensből álló vektorokat tartalmaznak. Konyhanyelven $W$~$\Rn$~altere ha ,,nem lehet kijutni belőle'', zárt.
	
	Érdemes megjegyezni, hogy az axiómarendszerünk kiegészíthető, pl.: semleges elem és inverz \emph{egyértelműen} létezik. Ez következik abból, hogy az összeadás itt kommutatív, azonban ezeknek a megértése könnyebb a diszkrét matematika csoportlemélettel foglalkozó részének elhangzása után.
	
	Fontosabb megállapítás azonban az, hogy ha $\lambda \in \R,~\b \in \Rn: ~\lambda\b=\0$, akkor $\lambda=0$ vagy $\b=\0$. Ez nagyon hasonlít a valós számok szorzatánál, ahol ha a szorzat egyenlő nullával, akkor a szorzat (legalább) egyik tagja is egyenlő nullával.
	
	Most egyszerre több vektorral fogunk foglalkozni, így bevezetjük a \textbf{vektorrendszer} (röviden \textbf{vr}) fogalmát. Egy vektorrendszer csupán annyiban különbözik egy halmaztól, hogy ugyanabból az elemből többet is tartalmazhat (míg egy halmazban egyszerre egy elem csak egyszer szerepelhet).
	
	DEFINÍCIÓ: Legyen $k\ge1$,~$\a_1, \a_2, \ldots , \a_k \in \Rn,~ \lambda_1, \lambda, \ldots , \lambda_k \in \R.$ Az $\a_1, \a_2, \ldots , \a_k$ \textbf{vr}. $\lambda_1, \lambda, \ldots , \lambda_k$ együtthatós \emph{lineáris kombinációja}:
	
	\[\l _1\a_1+\l _2\a _2+\dots+\l _k \a_k \in \Rn,\]
	
	\noindent Abban az esetben, ha  $\l_1=0,  \l_2=0, \ldots, \l_k=0$, akkor \emph{triviális} lineáris kombinációról beszélünk, vagyis
	\[0\a _1+0\a_2\dots+0\a _k.\]
	
	\noindent Értelemszerűen minden vektorrendszer \emph{triviális} lineáris kombinációja \0.
	
	
	DEFINÍCIÓ: {\tt Az} $\a _1, \a _2, \dots, \a _k \;  \Rn${ -beli vr}
	\emph{lineaárisan összefüggő} (rövidítve: Ö), { ha} 
	$\exists~ \l _1, \l _2, \dots, \l _k \in  \Rn$, { nem mind} $0$, 
	{ melyekre} $\l _1\a _1+\l _2 \a _2+\dots+\l _k\a_ k = \0$ (azaz, ha a
	vektorrendszernek létezik nullvektort adó { nemtriviális} lineáris
	kombinációja). 
		
	DEFINÍCIÓ: {\tt Az} $\a _1, \a _2, \dots, \a _k \;  \Rn${ -beli vr}
	\emph{lineárisan független}  (rövidítve: L), { ha} 
	$\{\m _1, \m _2, \dots, \m _k \in  \R ;~ 
	\m _1\a _1+\m _2 \a _2+\dots+\m_k\a_ k = \0$ $\Rightarrow $ $\m _1 =\m _2 = \dots = \m _k = 0\}$
	(azaz, ha a vektorrendszernek CSAK { a triviális} lineáris kombinációja ad
	nullvektort). 
	
	Vegyünk ezen definíciókra egy példát: legyen adott két $\R^2$-béli vektort, melynek \emph{valamilyen} lineáris kombinációja egyenlő $\0 \in \R^2$-val. Ha lineárisan függetlenek, akkor egy rövid számolással megállapíthajuk hogy mindkét együttható is egyenlő nullával:\\  ($\alpha, \beta \in \R$)
	
	\begin{center}
		$\alpha
	\begin{bmatrix}
		1\s\\
		0\s \\
	\end{bmatrix}+
	\beta
	\begin{bmatrix}
	0\s\\
	1\s \\
	\end{bmatrix}=
	\begin{bmatrix}
	0\s\\
	0\s \\
	\end{bmatrix}$
	\end{center}
	
	\noindent Ez két egyenletre bontható:
	
	\begin{center}
		$\left.\begin{matrix}
		1\alpha+0\beta = 0\s \\
		0\alpha+1\beta = 0\s \\
	\end{matrix}\right\} \Rightarrow
	\begin{matrix}
		\alpha=0\s\\
		\beta=0\s\\
	\end{matrix}$
	\end{center}
	
	\noindent Ezzel beláttuk hogy
	$\begin{bmatrix}
		1\s\\
		0\s \\
	\end{bmatrix}$ és$
	\begin{bmatrix}
		0\s\\
		1\s \\
	\end{bmatrix}$ L, hisz lineáris kombinációjuk csak akkor lehet \0-val egyenlő, ha a lineáris kombinációjuk triviális. Ha olyan eredményt kaptunk volna (nyilván más vektorokkal vett \0 eredményű lineáris kombinációnál) mint pl. $\alpha=-\beta$, akkor leolvasható hogy mindaddig amíg $\alpha~\beta$ ellentetje, az eredmény \0 lesz, így lineárisan összefüggők (pl. 
	$\begin{bmatrix}
		1\s\\
		0\s \\
	\end{bmatrix}$ és 
	$\begin{bmatrix}
	1\s\\
	0\s \\
	\end{bmatrix}$-ből ugyanezzel a módszerrel levezethető hogy $\alpha=-\beta$).
	
	Fontos megjegyezni: attól még hogy egy vr. triviálisan elő \emph{tudja} állítani a nullvektort, még nem lesz feltétlenül L. Így $\m_1=\ldots=\m_k=0$ és $\a_1,\ldots, \a_k \in \Rn$~esetén  \begin{center}
		$\m_1\a_1+\ldots+\m_k\a_k=\0 \stackrel{\text{\textit{ált.}}}{\not\Rightarrow}\a_1,\ldots, \a_k $~L,
	\end{center}
	\noindent hisz minden triviális lineáris kombináció \0-t eredményez.
	
	DEFINÍCIÓ: $V\leqq\Rn$~esetén legyen $\b _1, \dots, \b _k \in
	V$. A $\b
	_1, \dots, \b _k$ BÁZIS (rövidítve: B) $V$-ben, ha L és a $V$
	minden vektorát elõállítják lineáris kombinációjukként.  
	\medskip
	
	\noindent Érdemes tisztázni pár dolgot előre:
	\begin{compactitem}
		\item A többesszám hiány csalóka (bázisok helyett bázis), de itt egy \emph{vektorrendszer} alkot bázist, nem (feltétlenül) egy vektor.
		\item Azt hogy mindent előállít, mondhatjuk úgy is, hogy veszünk egy tetszőleges vektort $V$-ből, és ha $\b _1, \dots, \b _k$ B, akkor ez az előbb említett vektor mindenképpen felírható $\b _1, \dots, \b _k$ lineáris kombinációjával.
	\end{compactitem}
	Egy jó magyarázat arra, mitől lesz bázis a bázis: Vegyünk magunk elé 3 darab (lehetőleg nem 0 hosszúságú) tollat, és ebből kettőt rekjunk magunk elé az asztalunkra úgy, hogy merőlegesek legyeneke egymásra. Ekkor olyasvalami kreáltunk, amely a derékszögő koordinátarendszerhez hasonlít: $x$ tengely az egyik, $y$ tengely a másik toll.
	
	\begin{center}
		\emph{Kérdés:} Bázist alkot-e ez a két vektor (toll) az asztal síkjában? 
	\end{center}
	
	Válasz: Igen! Hiszen a harmadik tollat bárhova is helyezzük,  a másik két toll ,,megnyújtásával/zsugorításával, megfordításával és összeadásával'' megkaphatjuk azt. Mégha az asztal síkján egy millió kilométer hosszú vektort is rakunk, azt is ki tudjuk fejezni az eredeti két toll lineáris kombinációjával.\\
	Továbbá megállapítható, hogy a két ,,vektor'' csak triviális lineáris kombinációval állítják elő a nullvektort: ha bármely mász számmal szoroznánk bármelyiket is, a végeredmény garantáltan nem lenne \0.
	
	\medskip
	Most kicsit forgassuk el az \emph{egyik} tollat, úgy hogy ne legyen már derékszög de még ne is legyenek párhuzamosak egymással.
	
	\begin{center}
		\emph{Kérdés:} Bázist alkot-e ez a két vektor az asztal síkjában?
	\end{center}
	
	Válasz: Igen! Hisz az előbb említett tulajdonságot nem vesztettük el, \emph{igaz}, ugyanazt a vektort más lineáris kombinációval fogunk csak tudni felírni, de továbbra is megtehetjük, és ez a rendszer továbbra is L.
	
	Most növeljük meg a bezárt szögüket annyira, hogy a két toll egymással párhuzamos legyen, és ellenétes irányú.
	\begin{center}
		\emph{Kérdés:} Bázist alkot-e ez a két vektor az asztal síkjában?
	\end{center}
	
	\emph{Válasz:} Nem! Bárhogyan adjuk össze a két ,,vektort'', azok csak a saját vonalukban lévő vektorokat fogják tudni előállítani. Továbbá, ha úgy nyújtjuk és adjuk össze őket, hogy a két ,,vektor'' hossza egyenlő, de irányuk ellentétes, akkor nullvektort kapunk, vagyis lineárisan összefüggő is a rendszer.
	
	Milyen lehet egy vektorrendszer, melynél egyértelmű, vagy \emph{triviális} az hogy bázist alkot?
	Példaképp lehet akár ez is:
	
	{\centering 
		$\e _1=\begin{bmatrix}
		1\s \\
		0\s \\
		\vdots\\
		0\s \\
		\end{bmatrix},
		\e _2=\begin{bmatrix}
		0\s \\
		1\s \\
		\vdots\\
		0\s \\
		\end{bmatrix}, \dots ,
		\e _n=\begin{bmatrix}
		0\s \\
		0\s \\
		\vdots\\
		1\s \\
		\end{bmatrix} \in \Rn$.
		\par}
	
	\noindent Itt könnyen látható, hogy $\e_1, \e_2, \ldots, \e_n$ L, továbbá minden vektort elő fog tudni állítani $\Rn$-ben. Ezt a vektorrendszert \emph{triviális bázisnak} nevezzük, és általában $\e_1, \e_2, \ldots, \e_n$ módon nevezzük elemeit.
	
	Szánjunk 1 percet arra, hogy belássuk hogy ez valóban L:
	
	{\centering
		$\e _1=\begin{bmatrix}
	1\s \\
	0\s \\
	\vdots\\
	0\s \\
	\end{bmatrix},
	\e _2=\begin{bmatrix}
	0\s \\
	1\s \\
	\vdots\\
	0\s \\
	\end{bmatrix}, \dots ,
	\e _n=\begin{bmatrix}
	0\s \\
	0\s \\
	\vdots\\
	1\s \\
	\end{bmatrix} \in \Rn $ és $\l_1, \l_2, \ldots \l_n \in \R$ esetén:
	
	$\l_1\begin{bmatrix}
	1\s \\
	0\s \\
	\vdots\\
	0\s \\
	\end{bmatrix}+
	\l_2\begin{bmatrix}
	0\s \\
	1\s \\
	\vdots\\
	0\s \\
	\end{bmatrix}+ \dots+
	\l_n\begin{bmatrix}
	0\s \\
	0\s \\
	\vdots\\
	1\s \\
	\end{bmatrix}=
	\begin{bmatrix}
	0\s \\
	0\s \\
	\vdots\\
	0\s \\
	\end{bmatrix} \Leftrightarrow$
	
	$\begin{bmatrix}
	\l_1\s \\
	0\s \\
	\vdots\\
	0\s \\
	\end{bmatrix}+
	\begin{bmatrix}
	0\s \\
	\l_2\s \\
	\vdots\\
	0\s \\
	\end{bmatrix}+ \dots+
	\begin{bmatrix}
	0\s \\
	0\s \\
	\vdots\\
	\l_n\s \\
	\end{bmatrix}=
	\begin{bmatrix}
	0\s \\
	0\s \\
	\vdots\\
	0\s \\
	\end{bmatrix} \quad \Leftrightarrow$
	
	$\begin{bmatrix}
	\l_1\s \\
	\l_2\s \\
	\vdots\\
	\l_n\s \\
	\end{bmatrix}=
	\begin{bmatrix}
	0\s \\
	0\s \\
	\vdots\\
	0\s \\
	\end{bmatrix} \Leftrightarrow $
	
	\medskip
	$\l_1=\l_2=\ldots=\l_n=0.$
	\par}
	
	Ezzel beláttuk hogy a $\e_1, \ldots ,\e_n$ lineárisan független (és nyilvánvalóan bázis is, ezeket hamarosan belátjuk az elkövetkezendő előadásokban).
	
	\pagebreak
	\begin{center}
		ÖSSZEFOGLALÁS:
	\end{center}
	
	
		 \noindent A középsikolai vektorokkal kényelmes számolni, így egy ehhez hasonló struktúrát alkottunk.\\
		 Ez a struktúra a vektortér névre hallgat, 2 művelettel rendelkezik: összeadás és skaláris szorzás, melynek 10 tulajdonságát kigyűjtöttük.\\
		 Definiáltuk a vektortér halmazának speciális részhalmazait is, az altereket.\\
		 Definiáltunk a lineáris kombinációt, lineáris összefüggőséget és függetlenséget.\\
		 Definiáltuk a bázist, és egy általános példát is láttunk rá: a triviális bázist.
	
\end{document}
