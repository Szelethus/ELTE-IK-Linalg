\documentclass[a4paper,11.5pt]{article}
\usepackage[normalem]{ulem}
\usepackage[utf8]{inputenc}
\usepackage[T1]{fontenc}
\PassOptionsToPackage{defaults=hu-min}{magyar.ldf}
\usepackage{pgfplots}
\pgfplotsset{compat=1.10}
\usepgfplotslibrary{fillbetween}
\usepackage[magyar]{babel}
\usepackage{amsmath, amsthm,amssymb,paralist,array, ellipsis, graphicx, float, bigints,tikz}
%\usepackage{marvosym}

\makeatletter
\renewcommand*{\mathellipsis}{%
	\mathinner{%
		\kern\ellipsisbeforegap%
		{\ldotp}\kern\ellipsisgap
		{\ldotp}\kern\ellipsisgap%
		{\ldotp}\kern\ellipsisaftergap%
	}%
}
\renewcommand*{\dotsb@}{%
	\mathinner{%
		\kern\ellipsisbeforegap%
		{\cdotp}\kern\ellipsisgap%
		{\cdotp}\kern\ellipsisgap%
		{\cdotp}\kern\ellipsisaftergap%
	}%
}
\renewcommand*{\@cdots}{%
	\mathinner{%
		\kern\ellipsisbeforegap%
		{\cdotp}\kern\ellipsisgap%
		{\cdotp}\kern\ellipsisgap%
		{\cdotp}\kern\ellipsisaftergap%
	}%
}
\renewcommand*{\ellipsis@default}{%
	\ellipsis@before
	\kern\ellipsisbeforegap
	.\kern\ellipsisgap
	.\kern\ellipsisgap
	.\kern\ellipsisgap
	\ellipsis@after\relax}
\renewcommand*{\ellipsis@centered}{%
	\ellipsis@before
	\kern\ellipsisbeforegap
	.\kern\ellipsisgap
	.\kern\ellipsisgap
	.\kern\ellipsisaftergap
	\ellipsis@after\relax}
\AtBeginDocument{%
	\DeclareRobustCommand*{\dots}{%
		\ifmmode\@xp\mdots@\else\@xp\textellipsis\fi}}
\def\ellipsisgap{.1em}
\def\ellipsisbeforegap{.05em}
\def\ellipsisaftergap{.05em}
\makeatother

\usepackage{hyperref}
\hypersetup{
	colorlinks = true	
}

\DeclareMathOperator{\Int}{int}
\DeclareMathOperator{\tg}{tg}
\DeclareMathOperator{\ctg}{ctg}
\DeclareMathOperator{\sign}{sign}
\DeclareMathOperator{\Th}{th}
\DeclareMathOperator{\sh}{sh}
\DeclareMathOperator{\ch}{ch}
\DeclareMathOperator{\arsh}{arsh}
\DeclareMathOperator{\arch}{arch}
\DeclareMathOperator{\arth}{arth}
\DeclareMathOperator{\arcth}{arcth}
\DeclareMathOperator{\grad}{grad}
\DeclareMathOperator{\arc}{arc}
\DeclareMathOperator{\arctg}{arc tg}
\DeclareMathOperator{\arcctg}{arc ctg}
\newcommand{\norm}[1]{\left\lVert#1\right\rVert}

\begin{document}
	%%%%%%%%%%%RÖVIDÍTÉSEK%%%%%%%%%%
	\setlength\parindent{0pt}
	\def\a{\textbf{a}}
	\def\b{\textbf{b}}
	\def\N{\hskip 10 true mm}
	\def\a{\textbf{a}}
	\def\b{\textbf{b}}
	\def\c{\textbf{c}}
	\def\d{\textbf{d}}
	\def\x{\textbf{x}}
	\def\e{\textbf{e}}
	\def\gg{$\gamma$}
	\def\vi{\textbf{i}}
	\def\jj{\textbf{j}}
	\def\kk{\textbf{k}}
	\def\fh{\overrightarrow}
	\def\l{\lambda}
	\def\m{\mu}
	\def\v{\textbf{v}}
	\def\0{\textbf{0}}
	\def\s{\hspace{0.2mm}\vphantom{\beta}}
	\def\Z{\mathbb{Z}}
	\def\Q{\mathbb{Q}}
	\def\R{\mathbb{R}}
	\def\C{\mathbb{C}}
	\def\N{\mathbb{N}}
	\def\Rn{\mathbb{R}^{n}}
	\def\Ra{\overline{\mathbb{R}}}
	\def\sume{\displaystyle\sum_{n=1}^{+\infty}}
	\def\sumn{\displaystyle\sum_{n=0}^{+\infty}}
	\def\biz{\emph{Bizonyítás:\ }}
	\def\narrow{\underset{n\rightarrow+\infty}{\longrightarrow}}
	\def\limn{\displaystyle\lim_{n\to +\infty}}
	%	\def\definition{\textbf{Definíció:\ }}
	%	\def\theorem{\textbf{Tétel:\ }}
	%\def\note{\emph{Megjegyzés:\ }}
	%\def\example{\textbf{Példa:\ }} 
	
	\theoremstyle{definition}
	\newtheorem{theorem}{Tétel}[subsubsection]
	
	\theoremstyle{definition}
	\newtheorem{definition}[theorem]{Definíció}
	\newtheorem{example}[theorem]{Példa}
	\newtheorem{exercise}[theorem]{Házi feladat}
	\newtheorem{note}[theorem]{Megjegyzés}
	\newtheorem{task}[theorem]{Feladat}
	\newtheorem{revision}[theorem]{Emlékeztető}
	%%%%%%%%%%%%%%%%%%%%%%%%%%%%%%%%%
	
	
	\begin{flushright}
		
		Husi kidolgozása (\LaTeX)
		
	\end{flushright}
	
	\begin{center}
		
		Lineáris algebra (A, B, C)\\
		
		2. előadás
		
	\end{center}
	
	Az előző előadásban említettem, hogy szó lesz a különböző matematikai jelekről, és arról, hogyan kell helyesen kiolvasni egy tételt:
	
	\smallskip
	$\Leftarrow, \Rightarrow$: Implikáció, paraszti nyelven ,,Akkor''. A jelentés egyik oldalából következik a másik.
	
	Pl.: Egy síkidom négyzet. $\Rightarrow$ Ez a síkidom négyszög.
	
	\textit{Minden négyzet négyszög, de nem minden négyszög négyzet.}
	
	\smallskip
	$\Leftrightarrow$: Ekvivalencia, gyakorlatilag a logikai kijelentések egyenlősége.
	
	Pl.: $a \in \Z$ páros $\Leftrightarrow$ $a$ osztható 2-vel.
	
	\textit{Minden páros szám osztható 2vel, és minden 2vel osztható szám páros.}
	
	\smallskip
	És most következzen az első komolyabb tételünk:
	\begin{theorem}
	$\b_1, \b_2, \ldots , \b_k$\ B\ $V(\leqq \R^n)$-ben, $\a \in V \Rightarrow \exists! \alpha_1, \alpha_2, \ldots, \alpha_k \in \R :$
	$\a=\alpha_1\b_1+\alpha_2\b_2+\ldots+\alpha_k\b_k$.
	\end{theorem}
	
	\textit{$\b_1, \b_2, \ldots , \b_k$ {legyen bázis $\R^n$ egy alterében, melynek neve legyen $V$. Vegyünk még egy vektort $V$-ből, mely legyen }$\a$. {Ekkor egyértelműen létezik $\alpha_1, \alpha_2, \ldots, \alpha_k$ valós számok melyekkel elő lehet állítani }\a\ {vektort ebből a bázisból}}.\\
	
	   Bizonyítás: Két dolgot kell belátni: a létezés, és az egyértelműség. Ebből az első adódik abból hogy,  $\b_1, \b_2, \ldots , \b_k$ bázis. Az egyértelműség egy kis trükkel belátható: Legyen
	\begin{center}
		$\a=\alpha_1'\b_1+\alpha_2'\b_2+\ldots+\alpha_k'\b_k$ \quad ($\alpha_1',\ldots,\alpha_k'\in\R$),
		
		$\a=\alpha_1''\b_1+\alpha_2''\b_2+\ldots+\alpha_k''\b_k$ \quad ($\alpha_1'',\ldots,\alpha_k''\in\R$).
	\end{center}
	
	A tétel szerint \a-t $\b_1,\ldots\b_k$-val csak ,,egyféleképpen'' tudjuk előállítani, azaz a fenti két egyenletnek azonosnak kell lennie. Ez a mi szempontunkból azt jelenti, hogy minden egy vesszős együttható egyenlő a két vesszős megfelelőjével. Ezt formálisan így is kifejezhetjük: $\alpha_i'=\alpha_i''\quad (i=1\ldots k)$. Vonjuk ki a két egyenletet egymásból:
	
	\begin{center}
		$\0=(\alpha_1'-\alpha_1'')\b_1+(\alpha_2'-\alpha_2'')\b_2+\ldots+(\alpha_k'-\alpha_2'')\b_k$
	\end{center}
	
	   Mivel $\b_1, \b_2, \ldots , \b_k$ bázis, így L is, és egy L vektorrendszer csak triviálisan tudja előállítani \0-t, így minden zárójeles tagnak egyenlőnek kell lennie 0-val:
	
	\begin{center}
		$\begin{matrix}
			(\alpha_1'-\alpha_1'')=0&\Rightarrow&\alpha_1'=\alpha_1''\\
			(\alpha_2'-\alpha_2'')=0&\Rightarrow&\alpha_2'=\alpha_2''\\
			\vdots&&\vdots\\
			(\alpha_k'-\alpha_k'')=0&\Rightarrow&\alpha_k'=\alpha_k''\\
		\end{matrix}$
	\end{center}
	
	Ezzel a tételt bebizonyítotuk. $\blacksquare$ 
	
	A fenti bizonyítás a hivatalos jegyzetben 3 darab sor -- itt a cél az hogy elsajátítsuk a tétel helyes kiolvasásának készségét, és megértsük a tételek bizonyításának gondolatmenetét. Vizsgáljuk meg ennek a tételnek a megfordítását:\\
	
	\begin{theorem}
		Ha $V\leqq \Rn$ ~és $ \b_1, \b_2, \ldots , \b_k \in V$ olyan, hogy $\forall \a \in V\\
		\exists! \alpha_1, \alpha_2, \ldots, \alpha_k \in \R: \a =\alpha_1\b_1+\alpha_2\b_2+\ldots+\alpha_k\b_k \Rightarrow \b_1, \b_2, \ldots , \b_k$ bázis $V$-ben.
	\end{theorem}
	
	\emph{Legyen $V$ $\Rn$ egy altere, és legyen $ \b_1, \b_2, \ldots , \b_k$ ezen altérnek elemei. Ha minden $V$-béli $\a$ vektorra teljesül, hogy $ \b_1, \b_2, \ldots , \b_k$ egyértelműen elő tudja állítani lineáris kombinációval (azaz egyértelműen létezik minden \a-hoz olyan $\alpha_1, \alpha_2, \ldots, \alpha_k$ valós számok, hogy $\a=\alpha_1\b_1+\alpha_2\b_2+\ldots+\alpha_k\b_k$), akkor $ \b_1, \b_2, \ldots , \b_k$ bázis.}
	
	\smallskip
	Bizonyítás: Ismét 2 dolgot kell belátnunk: Ha $ \b_1, \b_2, \ldots , \b_k$ bázis, akkor egyrészt minden $V$-beli vektort elő tud állítani egyértelműen lineáris kombinációval, másrészt L is. Az előállítás következik a tétel feltételéből, L pedig abból, hogyha minden vektort egyértelműen állít elő, akkor ez \0-ra is teljesül, és mivel $\0=0\b_1+0\b_2+\ldots+0\b_k$, az egyértelműség miatt más előállítása \0-nak e vektorrendszerből nem lehet. Ezzel bebizonyítottuk a tételt. $\blacksquare$
	
	A következő tételünk nagyon hasznos lesz a további munkánk folyamán, számtalanszor fogjuk alkalmazni. Még ha bonyolultnak is tűnik, számos későbbi tétel megértéséhez kulcsfontosságú lehet. Továbbá a ez egy un. \textit{konstruktív bizonyítás}, azaz nemcsak a tétel helyességét látjuk be, azt is megmutatjuk, hogy hogyan lehet magát az algoritmust végrehajtani.
	
	\begin{theorem}
		(ELEMI BÁZIS TRANSZFORMÁCIÓ): Legyen $V \leqq \Rn, \b_1, \ldots, \b_k$ bázis $V$-ben, $\a \in V$, $\a = \alpha_1\b_1+\ldots+\alpha_k\b_k$, $i$ rögzített, $1\leqq i \leqq k$. Ekkor:
		\[ \b_1,\ldots,\b_{i-1},\a,\b_{i+1},\ldots,\b_k \text{ bázis } V \text{-ben} \quad\Leftrightarrow \quad \alpha_i\not=0. \]
	\end{theorem}\textit{Legyenek adottak az alábbi feltételek:
	\begin{compactitem}
		\item $V~\Rn$\ altere,
		\item $\b_1, \ldots, \b_k$ bázis $V$-ben,
		\item \a\ $\in V$, és $\b_1, \ldots, \b_k$-vel így írható fel: $\a = \alpha_1\b_1+\ldots+\alpha_k\b_k$,
		\item Adott nekünk egy $i$ egész számunk, melyre teljesül hogy $1\leqq i \leqq k$.
	\end{compactitem}
	Ekkor ha $\b_1, \ldots, \b_k$ bázis $i$-edik elemét kicseréljük \a-ra, és továbbra is bázist alkot a kapott vektorrendszer, akkor $\alpha_i$ (\a\ vektor $\b_1, \ldots, \b_k$ bázis szerinti lineáris kombinációban) nem nulla, és amennyiben $\alpha_i$ nem nulla, $\b_1,\ldots,\b_{i-1},\a,\b_{i+1},\ldots,\b_k$ bázis.}\\

	\hspace*{-2cm}
	\begin{tikzpicture}
		\draw[help lines, color=gray!30, dashed] (-0.9,-0.9) grid (3.9,3.9);
		\draw[->,ultra thick] (-1,0)--(4,0) node[right]{$x$};
		\draw[->,ultra thick] (0,-1)--(0,4) node[above]{$y$};
		
		\draw[->, thick] (1,1) -- node [left = 1mm] {$\b_1$} (2,3);
		\draw[->, thick] (1,1) -- node [below = 1mm] {$\b_2$} (3,1);
	\end{tikzpicture}
	\begin{tikzpicture}
		\draw[help lines, color=gray!30, dashed] (-0.9,-0.9) grid (3.9,3.9);
		\draw[->,ultra thick] (-1,0)--(4,0) node[right]{$x$};
		\draw[->,ultra thick] (0,-1)--(0,4) node[above]{$y$};
		
		\draw[->, thick] (1,1) -- node [left = 1mm] {$\b_1$} (2,3);
		\draw[->, thick] (1,1) -- node [below = 1mm] {$\a$} (2,2);
	\end{tikzpicture}
	\begin{tikzpicture}
		\draw[help lines, color=gray!30, dashed] (-0.9,-0.9) grid (3.9,3.9);
		\draw[->,ultra thick] (-1,0)--(4,0) node[right]{$x$};
		\draw[->,ultra thick] (0,-1)--(0,4) node[above]{$y$};
		
		\draw[->, thick] (1,1) -- node [left = 1mm] {$\b_1$} (2,3);
		\draw[->, thick] (1,1) -- node [below = 1mm] {$\a$} (0.5,0);
		\draw (0.9,1.1) -- (1.1,0.9);
	\end{tikzpicture}
	
	{\centering Az első képen látható egy kételemű bázis, $\b_1,\b_2$. Legyen $i=2$ (azaz rögzítjük $i$-t 2-re), és keressünk egy alkalmas $\a$-t, melyre $\a=\lambda_1\b_1+\lambda_2\b_2$ esetén $\lambda \not=0$. A második képen $\a=\frac{1}{2}\b_1+\frac{1}{4}\b_2$, így kicserélhetjük $\b_2$-t \a-ra, úgy, hogy a vektorrendszer továbbra is bázist alkosson. A harmadik képen $\a=-\frac{1}{2}\b_1+0\b_2$, láthatóan így a csere nem vezet bázishoz. \par}	
	
	\medskip
	Bizonyítás: \fbox{$\Rightarrow$}: Dolgozzunk először az állítás második felével, azaz a tétel felételei alapján lássuk be, hogy \[\b_1,\ldots,\b_{i-1},\a,\b_{i+1},\ldots,\b_k \text{ bázis} \quad\Rightarrow \quad \alpha_i\not=0.\]
	Indirekt módon (azaz az állítás ellentettjéről lássuk be hogy nem igaz), tegyük fel, hogy $\alpha_i=0$, és $\b_1,\ldots,\b_{i-1},\a,\b_{i+1},\ldots,\b_k$ bázis. Így:
	
	\begin{center}
		$\begin{matrix}
			\a&=&\alpha_1\b_1+\ldots+\alpha_i\b_i+\ldots+\alpha_k\b_k\\
			\a&=&\alpha_1\b_1+\ldots+0+\ldots+\alpha_k\b_k\\
			\0&=&\alpha_1\b_1+\ldots+(-1)\a+\ldots+\alpha_k\b_k
		\end{matrix}$
	\end{center}
	
	Itt ellentmondásra jutottunk, hisz abból indultunk ki hogy $\b_1,\ldots,\b_{i-1},\a,\b_{i+1},\ldots,\b_k$ bázis, de egy \emph{nem triviális} lineáris kombinációval előállítják \0-t (hisz \a\  együtthatója, $-1\not=0)$. Ezzel a tétel ezen részét bebizonyítottuk.
	
	\medskip
	\fbox{$\Leftarrow$}: Most az állítás második fele:
	\[\b_1,\ldots,\b_{i-1},\a,\b_{i+1},\ldots,\b_k \text{ bázis} \quad\Leftarrow \quad \alpha_i\not=0.\]
	
	Mint mindig, 2 dolgot kell belátni ahhoz hogy egy vektorrendszer bázis legyen: lineáris függetlenség és hogy minden vektort elő tud állítani az alteren belül. Ne feledjük el, itt támaszkodhatunk arra hogy $\alpha_i\not=0$.
	
	\smallskip
	\emph{1. Mindent elő tud állítani:} Legyen $\x \in V$ tetszőleges, mutassuk meg, hogy $\b_1,\ldots,\b_{i-1},\a,\b_{i+1},\ldots,\b_k$ vektorrendszerrel előállítható!
	
	\begin{center}
		$\begin{gathered}
			\x=x_1\b_1+\ldots+x_i\b_i+\ldots+ x_k\b_k,\quad (x_1,\ldots x_k\in\R)\\
			\a = \alpha_1\b_1+\ldots+\alpha_i\b_i+\ldots+\alpha_k\b_k.\\
		\end{gathered}$
	\end{center}
	
	Kiemelve a második egyenletből $\b_i$-t (ezt megtehetjük, hisz a tétel feltételeiben szerepel hogy $\alpha_i\not=0$):
		
	\begin{center}
		$\begin{gathered}
			 \b_i=-\frac{\alpha_1}{\alpha_i}\b_1-\ldots+\frac{1}{\alpha_i} \a-\ldots-\frac{\alpha_k}{\alpha_i}\b_k
		\end{gathered}$
	\end{center}
	
	Behelyettesítve az első egyenletbe:
	
	\begin{center}
		$\begin{gathered}
			\x=x_1\b_1+\ldots+x_i\left(-\frac{\alpha_1}{\alpha_i}\b_1-\ldots+\frac{1}{\alpha_i} \a-\ldots-\frac{\alpha_k}{\alpha_i}\right)+\ldots+ x_k\b_k\\
			\x=\left(x_1-\frac{x_i}{\alpha_i}\alpha_1\right)\b_1+\ldots+\frac{x_i}{\alpha_i} \a+\ldots+\left(x_k-\frac{x_i}{\alpha_i}\alpha_k\right)\b_k.
		\end{gathered}$
	\end{center}
	
	Ezzel beláttuk, hogy \x-et nem csak $\b_1,\ldots,\b_{i-1},\b_i,\b_{i+1},\ldots,\b_k$ bázissal, de $\b_1,\ldots,\b_{i-1},\a,\b_{i+1},\ldots,\b_k$ vektorrendszerrel is előállítható, sőt, még azt is láttuk, hogyan változnak majd az együtthatók az \x-et adó lineáris kombinációban egy ilyen vektorcsere után. Ha
	\begin{center}
		$\x=x_1\b_1+\ldots+x_i\b_i+\ldots+ x_k\b_k,\quad (x_1,\ldots x_k\in\R),$\\
		$\x=y_1\b_1+\ldots+y_i\a+\ldots+ y_k\b_k,\quad (y_1,\ldots y_k\in\R),$
	\end{center}
	azaz \x\ ha az eredeti bázissal $x_1,\ldots,x_k$ együtthatókkal állítható elő, és $\b_1,\ldots,\b_{i-1},\a,\b_{i+1},\ldots,\b_k$ vektorrendszerrel meg $y_1,\ldots,y_k$ együtthatókkal akkor
	\begin{center}
		$\displaystyle y_i=\frac{x_i}{\alpha_i},$ \quad $t\not=i$ esetén pedig \quad $\displaystyle y_t=x_t-\frac{x_i}{\alpha_1}\alpha_i.$
	\end{center}
	Ez az első egyenletbe történő behelyettesítés után jól is látható.
	
	\smallskip
	\emph{2. Lineáris függetlenség:} Legyen $\m_1,\ldots,\mu_k \in \R$. Ekkor tekintsük $\b_1,\ldots,\a,\ldots,\b_k$ egy olyan lineáris kombinációját, mely \0-t eredményez. Akkor lesz ez a vektorrendszer L, ha rendre minden $\m$-s tag egyenlő 0-val, és más értéket nem vehet fel.
	\[\m_1\b_1+\ldots+\m_i\a_i+\ldots+\m_k\b_k=\0\]
	Még továbbra is teljesül, hogy $\a = \alpha_1\b_1+\ldots+\alpha_k\b_k$, így \a-t a fenti egyenletbe helyettesítve:
	
	\begin{center}
		$\m_1\b_1+\ldots+\m_i(\alpha_1\b_1+\ldots+\alpha_k\b_k)+\ldots+\m_k\b_k=\0$\\
		$(\m_1+\m_i\alpha_1)\b_1+\ldots+(\m_i\alpha_i)\b_i+\ldots+(\m_k+\m_i\alpha_k)\b_k=\0$
	\end{center}
	(Figyeljük meg, hogy az összeadás minden tagjában szerepel $\m_i$!) Mivel $\b_1,\ldots,\b_i,\ldots,\b_k$ bázis, így minden zárójeles tag nulla, így pont az összeadás $i$-edik eleméből kiolvasható hogy $\m_i\alpha_i=0,\ \alpha_i\not=0 \Rightarrow \mu_i=0$. Ebből máris következik a többi együttható értéke:
	
	\begin{center}
		$(m_1+0\alpha_1)\b_1+\ldots+(0\alpha_i)\b_i+\ldots+(\m_k+0\alpha_k)\b_k=\0$\\
		$\m_1\b_1+\ldots+0\b_i+\ldots+\m_k\b_k=\0$,\\
	\end{center}
	   így $\m_1=\m_2=\ldots=\m_k=0$. Ezzel beláttuk hogy $\b_1,\ldots,\a,\ldots,\b_k$ valóban bázis, és a tételt bebizonyítottuk. $\blacksquare$
	
	Ennek a tételnek a bizonyítási módszerét sokszor be lehet vetni, így érdemes alaposan elmélyedni benne. A következő előadásban megtanuljuk, hogyan lehet gyakorlatban kényelmesen alkalmazni ezt a tételt.
	
	\smallskip
	Most kerüljön tisztázásra a komponens/koordináta dolog.
	
	DEFINÍCIÓ: $V\leqq \Rn, \b_1,\ldots,\b_k$ bázis $V$-ben, $\a \in V, \a = \alpha_1\b_1+\ldots+\alpha_k\b_k$ esetén azt mondjuk, hogy az \a~koordinátái $\b_1,\ldots,\b_k$ bázisban $\alpha_1, \ldots, \alpha_k$. Jelölése:
	
	\begin{center}
		$[\a]_\b:=[\a]_{\b_1,\ldots,\b_k}:=
		\begin{bmatrix}
			\alpha_1\s\\
			\alpha_2\s\\
			\vdots\\
			\alpha_k\s
		\end{bmatrix}\in\R^k.$
	\end{center}
	
	Ez azt jelenti, hogyha valaha a koordináta kifejezést használjuk, meg kell adnunk azt is, milyen bázisban értelmezzük azt. Vegyünk egy gyors példát: legyen $\a=
		\begin{bmatrix}
			4\s\\
			5\s
		\end{bmatrix}$,\ 
		$\a=4
		\begin{bmatrix}
			1\s\\
			0\s
		\end{bmatrix}+5
		\begin{bmatrix}
			0\s\\
			1\s
		\end{bmatrix}$,  azaz 
		$\begin{bmatrix}
			1\s\\
			0\s\\
		\end{bmatrix},  
		\begin{bmatrix}
			0\s\\
			1\s\\
		\end{bmatrix}$ bázisban \a\ koordinátái 5 és 4, míg
	$\a=2
	\begin{bmatrix}
		2\s\\
		0\s
	\end{bmatrix}+1
	\begin{bmatrix}
		0\s\\
		5\s
	\end{bmatrix}$,
	   azaz
	$\begin{bmatrix}
		2\s\\
		0\s
	\end{bmatrix},
	\begin{bmatrix}
		0\s\\
		5\s
	\end{bmatrix}$
	   bázisban \a\ koordinátái 2 és 1.
	
	E definíció ismeretében írjunk fel egy, az elemi bázistranszformációhoz tartozó táblázatot, melyből kiolvasható hogyan változnak a koordináták a bázis megváltozásának következtében.
	
	\begin{center}
		\begin{tabular}{c|cc||c|cc}
			$\alpha_i\not=0$&\a&\x&&\a&\x\\
			\hline
			$\b_1$&$\alpha_1$&$x_1$&$\b_1$&0&$x_1-(x_i/\alpha_i)\alpha_1$\\
			\vdots&\vdots&\vdots&\vdots&\vdots&\vdots\\
			$\b_i$&$\alpha_i$&$x_i$&$\a$&1&$x_i/\alpha_i$\\
			\vdots&\vdots&\vdots&\vdots&\vdots&\vdots\\
			$\b_k$&$\alpha_k$&$x_k$&$\b_k$&0&$x_k-(x_i/\alpha_i)\alpha_k$\\
		\end{tabular}
	\end{center}
	
	   A táblázatból kiolvasható, hogy az \a\ vektor oszlopában $\b_1$-hez az $\alpha_1$, $\b_i$-hez $\alpha_i$, $\b_k$-hoz $\alpha_k$ együttható tartozik, azaz $\a=\alpha_i\b_1+\ldots+\alpha_k\b_k.$
	
	   A táblázat első fele (a dupla vonal előtt) az eredeti $\b_1,\ldots,\b_k$ bázisban mutatja a koordinátákat, a második fele az elemi bázistranszformáció után, melyben az $\b_i$-t \a-ra cseréljük. Az hogy miért épp ezek lesznek a koordináták, fentebb megfigyelhető az EBT bizonyításában.
	
	Most ideje letudni az asszociativitás bizonyítását véges sok vektor összegénél.
	\begin{theorem}
		Legyen $k\geqq 1, \a_1, \ldots, \a_k \in \R^n$. Ekkor az $\a_1+\a_2+\ldots+\a_k$ összeg tetszőlegesen zárójelezhető.
		
		\noindent Bizonyítás: Teljes indukcióval bizonyítunk. 1, illetve 2 tagra az állítás triviális, 3 tagra pedig már sikeresen beláttuk, hogy az összeg asszociatív (első előadás). Legyen hát az indukciós állításunk az, hogy a tétel teljesül k-nál kevesebb tagú összegekre.
		
		\noindent Vezessünk be egy segédváltozót: $s\in\N$ melyre teljesül hogy $1\leqq s \leqq k-1$. Tekintsük egy $k$ tagú összeg tetszőleges zárójelezését. $s$ segédváltozó használatával az összeg felírható így:
		\begin{center}
			$ (\a_1+\ldots+\a_s$ valamilyen zárójelezése)+$(\a_{s+1}+\a_2+\ldots+\a_k $ valamilyen zárójelezése)
		\end{center}
		\noindent
		Megállapítandó, hogy mind a két tag $k$-nál kevesebb tagból áll. Amennyiben $s=k-1$, akkor az összeg így néz ki:
		
		\begin{center}
			$ (\a_1+\ldots+\a_{k-1}$ valamilyen zárójelezése)+$\a_k $
		\end{center}
		
		\noindent Itt alkalmazható az indukciós feltevés, lévén a fenti összegben a zárójeles tag csak $k-1$ tagú, és a tételt (legalábbis $s=k-1$ esetre) bebizonyítottuk. Amennyiben $s\not=k-1$, akkor:
		\begin{center}
			$ \textcolor{blue}{\{}
			(\ldots((\a_1+\a_2)+\a_3)+\ldots)+\a_s\textcolor{blue}{\}}+ \textcolor{red}{[}\textcolor{green}{\{}\ldots((\a_{s+1}+\a_{s+2})+\a_{s+3})+\ldots\textcolor{green}{\}}+\a_{k}\textcolor{red}{]}$
		\end{center}
		
		\noindent (Itt a zárójelezés továbbra is tetszőleges, fentebb csak egy lehetőség látható!) Ez nagyon hasonlít az első előadásban I/3. pontban megemlített alakhoz: $\a+(\b+\c)$. Amennyiben a kék zárójel \a, a zöld \b\ és $\a_k$ végül \c, alkalmazhatjuk ezen a 3 tagú összegre már bebizonyított tételt, azaz:
		$$ \overbrace{\textcolor{blue}{\{}
		(\ldots((\a_1+\a_2)+\a_3)+\ldots)+\a_s\textcolor{blue}{\}}}^{\a}+ \textcolor{red}{[}\overbrace{\textcolor{green}{\{}\ldots((\a_{s+1}+\a_{s+2})+\a_{s+3})+\ldots\textcolor{green}{\}}}^{\b}+\overbrace{\a_{k}}^{\c}\textcolor{red}{]}$$
		\[ \a+(\b+\c)\quad \Leftrightarrow\quad (\a+\b)+\c \]
		\begin{center}
			$ \textcolor{red}{[}
			\overbrace{\textcolor{blue}{(}\ldots((\a_1+\a_2)+\a_3)+\ldots)+\a_s\textcolor{blue}{)}}^{\a}+ \overbrace{\textcolor{green}{\{}\ldots((\a_{s+1}+\a_{s+2})+\a_{s+3})+\ldots\textcolor{green}{\}}}^{\b}\textcolor{red}{]}+\overbrace{\a_{k}}^{\c}$
		\end{center}
		
		\noindent Mivel a piros zárójeles tag $k-1$ elemű, így alkalmazható rajta az indukciós állítás. Ezzel a tételt bebizonyítottuk. $\blacksquare$
	\end{theorem}
	
\end{document}
